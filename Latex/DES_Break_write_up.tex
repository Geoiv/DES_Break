\documentclass[11pt]{article}

\usepackage{setspace}
\usepackage[letterpaper,margin=1in]{geometry}
\usepackage[parfill]{parskip}

\title{DES - Security Through Obscurity}
\author{Brandon Crane, Monica Singh,George Wood}
\date{31 October 2017}

\begin{document}
\maketitle

\thispagestyle{empty}

\section{Individual Contributions}
All team members met up to work on the program together. George focused on creating the TCP connection between the computers. Brandon focused on the code to create the threads.
Monica focused on setting up her machine so that the pthread and socket libraries
could be executable.

\section{Description}
The C++ POSIX thread, or \texttt{pthread}, library was used for implementing the thread portion of the assignment. To create a connection between our machines we used the the C++ \texttt{sys/socket} and  \texttt{netinet/in}  libraries. The communication between the client and the server was created using IPv4 and TCP, with the socket set to using Port 3000. The program creates a connection between a server and twelve clients. Three of these clients will be our own machines, which was approved by Dr.Jaiswal, and the other nine will be AWS clients. The server will create a connection with the clients and assign a key space to each client. The server will then have twelve threads which are listening for responses from the clients. The clients will be running six threads. One of these threads will be a parent thread which will be listening for messages from the server. This thread will create five child threads and assign them with their key spaces. Once a child thread finds the key, it will send a message to the server. The server will then send a message to the parent thread of the other clients to inform them that the key has been found.


\section{Code (C++)}
\subsection{DESBreakConst.h}
\begin{verbatim}
#include <vector>
#include <string>
#include <cmath>


const int PORT_NO = 3000;
const int MAX_LINE = 4096;
const int CLIENT_COUNT = 5;

\end{verbatim}

\subsection{DESCipher.h}
\begin{verbatim}
/*
CS 455 - Project Part 1 - DES Implementation
Brandon Crane, Monica Singh, & George Wood
10/31/17
*/

#include <string>
#include <vector>

//Number of characters in each block of input text being processed
const short CHARS_IN_BLOCK = 8;
//Number of bits in a char variable
const short BITS_IN_CHAR = 8;
//Number of rounds the algorithm should operate for
const short ROUND_COUNT = 16;

class DESCipher
{
  private:
    //IP Table, takes in vector of size 64, permutes it, and modifies
    //left and right vectors passed by reference to give plaintext halves
    void initPerm(std::vector<bool> inputVector, std::vector<bool>& leftText,
      std::vector<bool>& rightText);

    //Inverse IP table, takes in vector of booleans of size 64 and returns
    //its permutation
    std::vector<bool> invInitPerm(std::vector<bool> inputVector);

    //PC1 Table, takes in vector of size 64, permutes it to size 56, and modifies
    //halves passed by reference to give initial key halves
    void pc1Perm(std::vector<bool> keyBits, std::vector<bool>& leftKey,
      std::vector<bool>& rightKey);

    //Key shift scheduler, takes in key halves and shifts them by a specified
    //number of bits based on the current round of encryption
    std::vector<bool> leftShiftSched(std::vector<bool> inputVector, short round);

    //PC2 Table, takes in key halves vectors of size 28 each, permutes them,
    //and returns a combined permuted key vector
    std::vector<bool> pc2Perm(std::vector<bool> leftKey, std::vector<bool> rightKey);

    //Takes in a vector of size 32, expands and permutes it with a hard-coded table,
    //and returns a vector of size 48
    std::vector<bool> eTablePerm(std::vector<bool> inputVector);

    //S Boxes, takes in vector of the right half of the plaintext of size 48
    //and returns a permuted and substituted vector of length 32
    std::vector<bool> sBoxSub(std::vector<bool> rightTextI);

    //Takes in vector of size 32, permutes it using a hard-coded table,
    //and returns another vector of size 32
    std::vector<bool> pTablePerm(std::vector<bool> inputVector);

  public:
    DESCipher();

    //Converts characters to bits and places them in a vector of bools
    static std::vector<bool> charsToBits(std::vector<char> inputVector);

    //Converts bits to chars and appends them to a string
    static std::string bitsToChars(std::vector<bool> inputVector);

    //Takes in a size 64 vector of booleans for both plaintext and the key
    //and performs DES encryption on them. Returns string of ciphertext for the
    //input block of plaintext
    std::string encrypt(std::vector<bool> plainTextBits, std::vector<bool> keyBits);

    //Takes in a size 64 vector of booleans for both plaintext and the key
    //and performs DES decryption on them. Returns string of plaintext for the
    //input block of ciphertext
    std::string decrypt(std::vector<bool> cipherTextBits, std::vector<bool> keyBits);
};
\end{verbatim}

\subsection{pthread_barrier.h}
\begin{verbatim}
#ifdef __APPLE__

#ifndef PTHREAD_BARRIER_H_
#define PTHREAD_BARRIER_H_

#include <pthread.h>
#include <errno.h>

typedef int pthread_barrierattr_t;
typedef struct
{
    pthread_mutex_t mutex;
    pthread_cond_t cond;
    int count;
    int tripCount;
} pthread_barrier_t;


int pthread_barrier_init(pthread_barrier_t *barrier, const pthread_barrierattr_t *attr, unsigned int count)
{
    if(count == 0)
    {
        errno = EINVAL;
        return -1;
    }
    if(pthread_mutex_init(&barrier->mutex, 0) < 0)
    {
        return -1;
    }
    if(pthread_cond_init(&barrier->cond, 0) < 0)
    {
        pthread_mutex_destroy(&barrier->mutex);
        return -1;
    }
    barrier->tripCount = count;
    barrier->count = 0;

    return 0;
}

int pthread_barrier_destroy(pthread_barrier_t *barrier)
{
    pthread_cond_destroy(&barrier->cond);
    pthread_mutex_destroy(&barrier->mutex);
    return 0;
}

int pthread_barrier_wait(pthread_barrier_t *barrier)
{
    pthread_mutex_lock(&barrier->mutex);
    ++(barrier->count);
    if(barrier->count >= barrier->tripCount)
    {
        barrier->count = 0;
        pthread_cond_broadcast(&barrier->cond);
        pthread_mutex_unlock(&barrier->mutex);
        return 1;
    }
    else
    {
        pthread_cond_wait(&barrier->cond, &(barrier->mutex));
        pthread_mutex_unlock(&barrier->mutex);
        return 0;
    }
}

#endif // PTHREAD_BARRIER_H_
#endif // __APPLE__

\end{verbatim}

\subsection{DESCipher.cpp}
\begin{verbatim}
/*
CS 455 - Project Part 1 - DES Implementation
Brandon Crane, Monica Singh, & George Wood
10/31/17
*/

#include <iostream>
#include <fstream>
#include <bitset>
#include <time.h>
#include "DESCipher.h"
using namespace std;

DESCipher::DESCipher()
{

}

//IP Table, takes in vector of size 64, permutes it, and modifies
//left and right vectors passed by reference to give plaintext halves
void DESCipher::initPerm(vector<bool> inputVector, vector<bool>& leftText,
  vector<bool>& rightText)
{
  short outputVectorSize = 64;
  //IP table stored as an array
  short initPerm[] = {58,50,42,34,26,18,10,2,
                      60,52,44,36,28,20,12,4,
                      62,54,46,38,30,22,14,6,
                      64,56,48,40,32,24,16,8,
                      57,49,41,33,25,17,9, 1,
                      59,51,43,35,27,19,11,3,
                      61,53,45,37,29,21,13,5,
                      63,55,47,39,31,23,15,7};
  //Vector to store permuted values
  vector<bool> permVec;
  //Pushes permuted values into permVec
  for(short i=0; i < outputVectorSize; i++)
  {
    permVec.push_back(inputVector.at(initPerm[i]-1));
  }
  //First half of permuted values pushed to the leftText vector
  for (unsigned short i = 0; i < permVec.size() / 2; i++)
  {
    leftText.push_back(permVec.at(i));
  }
  //Second half of permuted values pushed to the rightText vector
  for (unsigned short i = permVec.size() / 2; i < permVec.size(); i++)
  {
    rightText.push_back(permVec.at(i));
  }
}

//Inverse IP table, takes in vector of booleans of size 64 and returns
//its permutation
vector<bool> DESCipher::invInitPerm(vector<bool> inputVector)
{
  short outputVectorSize = 64;
  //Inverse IP tables stored as an array
  short invInitPerm[] = {40,8,48,16,56,24,64,32,
                         39,7,47,15,55,23,63,31,
                         38,6,46,14,54,22,62,30,
                         37,5,45,13,53,21,61,29,
                         36,4,44,12,52,20,60,28,
                         35,3,43,11,51,19,59,27,
                         34,2,42,10,50,18,58,26,
                         33,1,41,9,49,17,57,25};
  //Vector of bools to be returned
  vector<bool> outputVector;
  //Pushes permuted values to outputVector
  for(short i=0; i < outputVectorSize; i++)
  {
    outputVector.push_back(inputVector.at(invInitPerm[i]-1));
  }
  return outputVector;
}

//PC1 Table, takes in vector of size 64, permutes it to size 56, and modifies
//halves passed by reference to give initial key halves
void DESCipher::pc1Perm(vector<bool> keyBits, vector<bool>& leftKey,
  vector<bool>& rightKey)
{
  short pcOneSize = 56;
  //PC-1 table stored as array
  short pcOne[]= {57,49,41,33,25,17,9,
                  1, 58,50,42,34,26,18,
                  10,2, 59,51,43,35,27,
                  19,11,3, 60,52,44,36,
                  63,55,47,39,31,23,15,
                  7, 62,54,46,38,30,22,
                  14,6, 61,53,45,37,29,
                  21,13,5, 28,20,12,4};
  //Pushes first half of permuted values to leftKey
  for(short i = 0; i < pcOneSize / 2; i++)
  {
    leftKey.push_back(keyBits.at(pcOne[i]-1));
  }
  //Pushes second half of permuted values to rightKey
  for(short i = pcOneSize / 2; i < pcOneSize; i++)
  {
    rightKey.push_back(keyBits.at(pcOne[i]-1));
  }
}

//Key shift scheduler, takes in key halves and shifts them by a specified
//number of bits based on the current round of encryption
vector<bool> DESCipher::leftShiftSched(vector<bool> inputVector, short round)
{
  //Left shift schedule stored as array
  short schedule[] = {1,1,2,2,2,2,2,2,1,2,2,2,2,2,2,1};
  //Vector of bools to be returned
  vector<bool> outputVector;
  //The following two for loops perform the left shift
  for (unsigned short i = schedule[round]; i < inputVector.size(); i++ )
  {
    outputVector.push_back(inputVector.at(i));
  }
  for(short i = 0; i < schedule[round]; i++)
  {
    outputVector.push_back(inputVector.at(i));
  }
  return outputVector;
}

//PC2 Table, takes in key halves vectors of size 28 each, permutes them,
//and returns a combined permuted key vector
vector<bool> DESCipher::pc2Perm(vector<bool> leftKey, vector<bool> rightKey)
{
  //Key made up of the two half keys
  vector<bool> combinedKey = leftKey;
  combinedKey.insert(combinedKey.end(), rightKey.begin(), rightKey.end());

  short pcTwoSize = 48;
  //PC-2 table stored as array
  short pcTwo[]= {14,17,11,24,1, 5, 3, 28,
                  15,6, 21,10,23,19,12,4,
                  26,8, 16,7, 27,20,13,2,
                  41,52,31,37,47,55,30,40,
                  51,45,33,48,44,49,39,56,
                  34,53,46,42,50,36,29,32};
  //Vector of bools to be returned
  vector<bool> outputVector;
  //Pushes permuted values to outputVector
  for(short i=0; i<pcTwoSize; i++)
  {
    outputVector.push_back(combinedKey.at(pcTwo[i]-1));
  }
  return outputVector;
}

//Takes in a vector of size 32, expands and permutes it with a hard-coded table,
//and returns a vector of size 48
vector<bool> DESCipher::eTablePerm(vector<bool> inputVector)
{
  short outputVectorSize = 48;
  //E table stored as array
  short eTable[] = {32,1, 2, 3, 4, 5,
                    4, 5, 6, 7, 8, 9,
                    8, 9, 10,11,12,13,
                    12,13,14,15,16,17,
                    16,17,18,19,20,21,
                    20,21,22,23,24,25,
                    24,25,26,27,28,29,
                    28,29,30,31,32,1};
  //Vector of bools to be returned
  vector<bool> outputVector;
  //Pushes permuted values to outputVector
  for (short i = 0; i < outputVectorSize; i++)
  {
    outputVector.push_back(inputVector.at(eTable[i] - 1));
  }
  return outputVector;
}

//S Boxes, takes in vector of the right half of the plaintext of size 48
//and returns a permuted and substituted vector of length 32
vector<bool> DESCipher::sBoxSub(vector<bool> rightTextI)
{
  //Number of groups of to be applied to the S boxes
  short bitGroupCount = 8;
  //Number of bits in each group
  short bitsInGroup = 6;
  //Vector of S boxes
  vector<vector<vector<short>>> sBoxes;
  //Temp vectors used to create S boxes
  vector<short> tempRow;
  vector<vector<short>> tempBox;

  //S Box 1
  tempRow = {14, 4, 13, 1, 2, 15, 11, 8, 3, 10, 6, 12, 5, 9, 0, 7};
  tempBox.push_back(tempRow);
  tempRow = {0, 15, 7, 4, 14, 2, 13, 1, 10, 6, 12, 11, 9, 5, 3, 8};
  tempBox.push_back(tempRow);
  tempRow = {4, 1, 14, 8, 13, 6, 2, 11, 15, 12, 9, 7, 3, 10, 5, 0};
  tempBox.push_back(tempRow);
  tempRow = {15, 12, 8, 2, 4, 9, 1, 7, 5, 11, 3, 14, 10, 0, 6, 13};
  tempBox.push_back(tempRow);
  sBoxes.push_back(tempBox);
  tempBox.clear();

  ////S Box 2
  tempRow = {15, 1, 8, 14, 6, 11, 3, 4, 9, 7, 2, 13, 12, 0, 5, 10};
  tempBox.push_back(tempRow);
  tempRow = {3, 13, 4, 7, 15, 2, 8, 14, 12, 0, 1, 10, 6, 9, 11, 5};
  tempBox.push_back(tempRow);
  tempRow = {0, 14, 7, 11, 10, 4, 13, 1, 5, 8, 12, 6, 9, 3, 2, 15};
  tempBox.push_back(tempRow);
  tempRow = {13, 8, 10, 1, 3, 15, 4, 2, 11, 6, 7, 12, 0, 5, 14, 9};
  tempBox.push_back(tempRow);
  sBoxes.push_back(tempBox);
  tempBox.clear();

  //S Box 3
  tempRow = {10, 0, 9, 14, 6, 3, 15, 5, 1, 13, 12, 7, 11, 4, 2, 8};
  tempBox.push_back(tempRow);
  tempRow = {13, 7, 0, 9, 3, 4, 6, 10, 2, 8, 5, 14, 12, 11, 15, 1};
  tempBox.push_back(tempRow);
  tempRow = {13, 6, 4, 9, 8, 15, 3, 0, 11, 1, 2, 12, 5, 10, 14, 7};
  tempBox.push_back(tempRow);
  tempRow = {1, 10, 13, 0, 6, 9, 8, 7, 4, 15, 14, 3, 11, 5, 2, 12};
  tempBox.push_back(tempRow);
  sBoxes.push_back(tempBox);
  tempBox.clear();

  //S Box 4
  tempRow = {7, 13, 14, 3, 0, 6, 9, 10, 1, 2, 8, 5, 11, 12, 4, 15};
  tempBox.push_back(tempRow);
  tempRow = {13, 8, 11, 5, 6, 15, 0, 3, 4, 7, 2, 12, 1, 10, 14, 9};
  tempBox.push_back(tempRow);
  tempRow = {10, 6, 9, 0, 12, 11, 7, 13, 15, 1, 3, 14, 5, 2, 8, 4};
  tempBox.push_back(tempRow);
  tempRow = {3, 15, 0, 6, 10, 1, 13, 8, 9, 4, 5, 11, 12, 7, 2, 14};
  tempBox.push_back(tempRow);
  sBoxes.push_back(tempBox);
  tempBox.clear();

  //S Box 5
  tempRow = {2, 12, 4, 1, 7, 10, 11, 6, 8, 5, 3, 15, 13, 0, 14, 9};
  tempBox.push_back(tempRow);
  tempRow = {14, 11, 2, 12, 4, 7, 13, 1, 5, 0, 15, 10, 3, 9, 8, 6};
  tempBox.push_back(tempRow);
  tempRow = {4, 2, 1, 11, 10, 13, 7, 8, 15, 9, 12, 5, 6, 3, 0, 14};
  tempBox.push_back(tempRow);
  tempRow = {11, 8, 12, 7, 1, 14, 2, 13, 6, 15, 0, 9, 10, 4, 5, 3};
  tempBox.push_back(tempRow);
  sBoxes.push_back(tempBox);
  tempBox.clear();

  //S Box 6
  tempRow = {12, 1, 10, 15, 9, 2, 6, 8, 0, 13, 3, 4, 14, 7, 5, 11};
  tempBox.push_back(tempRow);
  tempRow = {10, 15, 4, 2, 7, 12, 9, 5, 6, 1, 13, 14, 0, 11, 3, 8};
  tempBox.push_back(tempRow);
  tempRow = {9, 14, 15, 5, 2, 8, 12, 3, 7, 0, 4, 10, 1, 13, 11, 6};
  tempBox.push_back(tempRow);
  tempRow = {4, 3, 2, 12, 9, 5, 15, 10, 11, 14, 1, 7, 6, 0, 8, 13};
  tempBox.push_back(tempRow);
  sBoxes.push_back(tempBox);
  tempBox.clear();

  //S Box 7
  tempRow = {4, 11, 2, 14, 15, 0, 8, 13, 3, 12, 9, 7, 5, 10, 6, 1};
  tempBox.push_back(tempRow);
  tempRow = {13, 0, 11, 7, 4, 9, 1, 10, 14, 3, 5, 12, 2, 15, 8, 6};
  tempBox.push_back(tempRow);
  tempRow = {1, 4, 11, 13, 12, 3, 7, 14, 10, 15, 6, 8, 0, 5, 9, 2};
  tempBox.push_back(tempRow);
  tempRow = {6, 11, 13, 8, 1, 4, 10, 7, 9, 5, 0, 15, 14, 2, 3, 12};
  tempBox.push_back(tempRow);
  sBoxes.push_back(tempBox);
  tempBox.clear();

  //S Box 8
  tempRow = {13, 2, 8, 4, 6, 15, 11, 1, 10, 9, 3, 14, 5, 0, 12, 7};
  tempBox.push_back(tempRow);
  tempRow = {1, 15, 13, 8, 10, 3, 7, 4, 12, 5, 6, 11, 0, 14, 9, 2};
  tempBox.push_back(tempRow);
  tempRow = {7, 11, 4, 1, 9, 12, 14, 2, 0, 6, 10, 13, 15, 3, 5, 8};
  tempBox.push_back(tempRow);
  tempRow = {15, 12, 8, 2, 4, 9, 1, 7, 5, 11, 3, 14, 10, 0, 6, 13};
  tempBox.push_back(tempRow);
  sBoxes.push_back(tempBox);

  //Vector to hold
  vector<vector<bool>> bitGroups;
  //Vector of bools to be returned
  vector<bool> outputVector;
  //Loads vectors of bools into bitGroups to represent each group of bits
  for(short i=0; i < bitGroupCount; i++)
  {
    vector<bool> temp;
    for(short j = 0; j < bitsInGroup; j++)
    {
      temp.push_back(rightTextI.at((i * bitsInGroup) + j));
    }
    bitGroups.push_back(temp);
  }

  //Applies an S box to each bit group
  for(short j = 0; j < bitGroupCount; j++)
  {
    //Number of bits that will be in each group after S box application
    short outputBitCount = 4;
    //String representing row of S box to be used
    string rowString;
    //String representing column of S box to be used
    string colString;

    //Gets row value in binary
    rowString.append(bitGroups.at(j).at(0) ? "1" : "0");
    rowString.append(bitGroups.at(j).at(bitsInGroup - 1) ? "1" : "0");
    //Gets column value in binary
    for(short i = 1; i <5; i++)
    {
      colString.append(bitGroups.at(j).at(i) ? "1" : "0");
    }
    //Bitsets to convert the row and column values to decimal
    bitset<2> row (rowString);
    bitset<4> col (colString);
    //Gets the value that each bit group will be replaced with
    bitset<4> newBinary(sBoxes.at(j).at(row.to_ulong()).at(col.to_ulong()));
    //Pushes the new binary value of the current group to outputVector
    for(short i = outputBitCount - 1; i >= 0; i--)
    {
      outputVector.push_back(newBinary[i]);
    }
  }
  return outputVector;
}

//Takes in vector of size 32, permutes it using a hard-coded table,
//and returns another vector of size 32
vector<bool> DESCipher::pTablePerm(vector<bool> inputVector)
{
  short outputVectorSize = 32;
  //P table stored as vector
  short pTable[] = {16,7, 20,21,29,12,28,17,
                    1, 15,23,26,5, 18,31,10,
                    2, 8, 24,14,32,27,3, 9,
                    19,13,30,6, 22,11,4,25};
  //Vector of bools to be returned
  vector<bool> outputVector;
  //Pushes permuted values to outputVector
  for (short i = 0; i < outputVectorSize; i++)
  {
    outputVector.push_back(inputVector.at(pTable[i] - 1));
  }
  return outputVector;
}

//Converts characters to bits and places them in a vector of bools
vector<bool> DESCipher::charsToBits(vector<char> inputVector)
{
  vector<bool> outputVector;
  for (short i = 0; i < CHARS_IN_BLOCK; i++)
  {
    bitset<BITS_IN_CHAR> temp(inputVector.at(i));
    for(short j = BITS_IN_CHAR - 1; j >= 0; j--)
    {
      outputVector.push_back(temp[j]);
    }
  }
  return outputVector;
}

//Converts bits to chars and appends them to a string
string DESCipher::bitsToChars(vector<bool> inputVector)
{
  //Output string of converted binary values to be returned
  string outputText = "";
  //Loops for each character in the block
  for (short i = 0; i < CHARS_IN_BLOCK; i++)
  {
    //String to hold the binary of the current character
    string tempString = "";
    //Adds bits for current character to string
    for (short j = 0; j < BITS_IN_CHAR; j++)
    {
      tempString += (inputVector.at((i * CHARS_IN_BLOCK) + j) ? "1" : "0");
    }
    //Puts bits in a bitset
    bitset<BITS_IN_CHAR> temp(tempString);
    //Adds converted character to the output text
    outputText += (char)temp.to_ulong();
  }
  return outputText;
}

//Takes in a size 64 vector of booleans for both plaintext and the key
//and performs DES encryption on them. Returns string of ciphertext for the
//input block of plaintext
string DESCipher::encrypt(vector<bool> plainTextBits, vector<bool> keyBits)
{
  //Output ciphertext to be returned
  string cipherText;

  //Left half of plaintext
  vector<bool> leftTextI;
  //Right half of plainText
  vector<bool> rightTextI;
  //Left half of key
  vector<bool> leftKey;
  //Right half of key
  vector<bool> rightKey;
  //Left half of text to be used next round
  vector<bool> leftTextIPlus1;
  //Key after PC-2 table
  vector<bool> permKey;
  //Applies IP table to plaintext and splits it into halves
  initPerm(plainTextBits, leftTextI, rightTextI);
  //Applies PC-1 table to key and splits it into halves
  pc1Perm(keyBits, leftKey, rightKey);

  //Loops for the specified number of rounds
  for (short i = 0; i < ROUND_COUNT; i++)
  {

    //Sets the left text half for the next round to current right half
    leftTextIPlus1 = rightTextI;
    //Right text is permuted and expanded to 48 bits
    rightTextI = eTablePerm(rightTextI);

    //Applies left shifts to key halves
    leftKey = leftShiftSched(leftKey, i);
    rightKey = leftShiftSched(rightKey, i);

    //Applies PC-2 table to the key halves and combines them
    permKey = pc2Perm(leftKey, rightKey);

    //Applies XOR on each bit of the right text half and the permuted key
    for (unsigned short j = 0; j < rightTextI.size(); j++)
    {
      rightTextI.at(j) = rightTextI.at(j)^permKey.at(j);
    }


    //Applies S boxes to text half
    rightTextI = sBoxSub(rightTextI);

    //Applies P table to text half
    rightTextI = pTablePerm(rightTextI);

    //Applies XOR on each bit of the mutated right text half and the left half
    for (unsigned short j = 0; j < rightTextI.size(); j++)
    {
      rightTextI.at(j) = rightTextI.at(j)^leftTextI.at(j);
    }

    //Sets the left text half for current round to the left half for next round
    leftTextI = leftTextIPlus1;

  }

  //Applies 32-bit swap to the left and right halves after all rounds
  rightTextI.insert(rightTextI.end(), leftTextI.begin(), leftTextI.end());

  //Applies inverse IP table to final output ciphertext
  rightTextI = invInitPerm(rightTextI);

  //Converts binary values back to characters
  cipherText = bitsToChars(rightTextI);

  return cipherText;
}

//Takes in a size 64 vector of booleans for both plaintext and the key
//and performs DES decryption on them. Returns string of plaintext for the
//input block of ciphertext
string DESCipher::decrypt(vector<bool> cipherTextBits, vector<bool> keyBits)
{
  //Output plaintext to be returned
  string plainText;

  //Left half of plaintext
  vector<bool> leftTextI;
  //Right half of plainText
  vector<bool> rightTextI;
  //Left half of key
  vector<bool> leftKey;
  //Right half of key
  vector<bool> rightKey;
  //Left half of text to be used next round
  vector<bool> leftTextIPlus1;
  //Key after PC-2 table
  vector<bool> permKey;
  //Applies IP table to plaintext and splits it into halves
  initPerm(cipherTextBits, leftTextI, rightTextI);
  //Applies PC-1 table to key and splits it into halves
  pc1Perm(keyBits, leftKey, rightKey);

  for (short i = ROUND_COUNT - 1; i >= 0; i--)
  {
    //Sets the shifted key halves (pre shift) equal to the original halves
    vector<bool> shiftedLeftKey = leftKey;
    vector<bool> shiftedRightKey = rightKey;

    //Sets the left text half for the next round to current right half
    leftTextIPlus1 = rightTextI;
    //Right text is permuted and expanded to 48 bits
    rightTextI = eTablePerm(rightTextI);

    //Applies left shifts to key halves
    for (short j = 0; j <= i; j++)
    {
      shiftedLeftKey = leftShiftSched(shiftedLeftKey, j);
      shiftedRightKey = leftShiftSched(shiftedRightKey, j);
    }

    //Applies PC-2 table to the key halves and combines them
    permKey = pc2Perm(shiftedLeftKey, shiftedRightKey);


    //Applies XOR on each bit of the right text half and the permuted key
    for (unsigned short j = 0; j < rightTextI.size(); j++)
    {
      rightTextI.at(j) = rightTextI.at(j)^permKey.at(j);
    }
    //Applies S boxes to text half
    rightTextI = sBoxSub(rightTextI);
    //Applies P table to text half
    rightTextI = pTablePerm(rightTextI);

    //Applies XOR on each bit of the mutated right text half and the left half
    for (unsigned short j = 0; j < rightTextI.size(); j++)
    {
      rightTextI.at(j) = rightTextI.at(j)^leftTextI.at(j);
    }
    //Sets the left text half for current round to the left half for next round
    leftTextI = leftTextIPlus1;
  }
  //Applies 32-bit swap to the left and right halves after all rounds
  rightTextI.insert(rightTextI.end(), leftTextI.begin(), leftTextI.end());
  //Applies inverse IP table to final output plaintext
  rightTextI = invInitPerm(rightTextI);

  //Converts binary values back to characters
  plainText = bitsToChars(rightTextI);
  return plainText;
}

\end{verbatim}

\subsection{server.cpp}
\begin{verbatim}
#include <iostream>
#include <sys/socket.h>
#include <netinet/in.h>
#include <unistd.h>
#include <cstring>
#include "DESBreakConsts.h"
#include "pthread_barrier.h"
using namespace std;

pthread_barrier_t threadBarrier;

struct thread_data
{
   int threadId;
   int clientFileDesc;
};

void *listenForClient(void * threadArg)
{
  struct thread_data *threadData;
  threadData = (struct thread_data *) threadArg;

  char recMsg[MAX_LINE];
  int receiveResult = recv(threadData->clientFileDesc, recMsg, MAX_LINE, 0);
  if (receiveResult == 0)
  {
    cout << "No available messages, or connection gracefully closed. " << endl;
  }
  else if (receiveResult == -1)
  {
    cout << "Error receiving message from client" << threadData->threadId << "." << endl;
    perror("Receiving message from client failed.");
    exit(EXIT_FAILURE);
  }
  else
  {
    cout << "Key found by thread " << threadData->threadId << "!" << endl;
    cout << "Key: " << recMsg << endl;
    pthread_barrier_wait(&threadBarrier);
  }
  pthread_exit(NULL);
}

//TODO is a new sockaddr_in required for each connection?
int main()
{
  pthread_barrier_init(&threadBarrier, NULL, 2);

  struct sockaddr_in servAddress;
  memset(&servAddress, 0, sizeof(servAddress));
  servAddress.sin_family = AF_INET;
  servAddress.sin_addr.s_addr = htonl(INADDR_ANY);
  servAddress.sin_port =  htons(PORT_NO);

  struct sockaddr_in clientAddress;
  socklen_t clientAddressLen;

  vector<int> serverSockFDs;
  vector<int> connectSockFDs;

  for (int i = 0; i < CLIENT_COUNT; i++)
  {
    // AF_INET = Use IPv4,  SOCK_STREAM = Use TCP
    int servFileDesc = socket(AF_INET, SOCK_STREAM, 0);
    if (servFileDesc == -1)
    {
      for (int j = 0; j < i; j++)
      {
        close(j);
      }
      cout << "Issue in starting commmunication with client #" << i << "." <<
        endl;
      perror("Server socket descriptor creation failed.");
      exit(EXIT_FAILURE);
    }

    /*
    int reuse = 1;
    if(setsockopt(servFileDesc, SOL_SOCKET, SO_REUSEADDR, &reuse, sizeof(reuse)) != 0);
    {
      perror("Socket option setting failed.");
      exit(EXIT_FAILURE);
    }
    */

    serverSockFDs.push_back(servFileDesc);
    cout << "Server file descriptor created: " << servFileDesc << endl;
    for(unsigned int i=0; i< serverSockFDs.size(); i++)
    {
      cout << "serverSockFDs.at(" << i << ") : " << serverSockFDs.at(i) << endl;
    }

    if (::bind(servFileDesc, (struct sockaddr *) &servAddress, sizeof(servAddress)) == -1)
    {
      for (int j = 0; j <= i; j++)
      {
        close(serverSockFDs.at(j));
        cout << "Closing... serverSockFDs.at(" << j << ") : " << serverSockFDs.at(j) << endl;
      }
      cout << "Issue in binding socket with client #" << i << "." <<
        endl;
      perror("Socket binding failed.");
      exit(EXIT_FAILURE);
    }

    if(listen(servFileDesc, 1) == -1)
    {
      for (int j = 0; j <= i; j++)
      {
        close(serverSockFDs.at(j));
      }
      cout << "Issue in starting commmunication with client #" << i << "." <<
        endl;
      perror("Socket listening failed.");
      exit(EXIT_FAILURE);
    }

    clientAddressLen = sizeof(clientAddress);
    int connectFileDesc = accept(servFileDesc,
      (struct sockaddr*) &clientAddress, &clientAddressLen);

    connectSockFDs.push_back(connectFileDesc);

    cout << "Connection request received..." << endl;
    const char* outgoingClientID = to_string(i).c_str();
    int sendResult = send(connectFileDesc, outgoingClientID,
      sizeof(outgoingClientID), 0);
    if (sendResult == -1)
    {
      for (int j = 0; j <= i; j++)
      {
        close(serverSockFDs.at(j));
        close(connectSockFDs.at(j));
      }
      cout << "Issue in commmunication with client #" << i << "." << endl;
      perror("Sending Client ID failed.");
      exit(EXIT_FAILURE);
    }
  }

  int listenerThreadCount = connectSockFDs.size();
  pthread_t threads[listenerThreadCount];
  struct thread_data threadData[listenerThreadCount];

  // Create a thread that will listen for the result of each client program.
  for (int i = 0; i < CLIENT_COUNT; i++)
  {
    threadData[i].threadId = i;
    threadData[i].clientFileDesc = connectSockFDs.at(i);

    int resultCode = pthread_create(&threads[i], NULL, listenForClient, (void *)&threadData[i]);
    if (resultCode)
    {
       cout << "Error: unable to create thread, " << resultCode << endl;
       exit(-1);
    }
  }

  pthread_barrier_wait(&threadBarrier);
  pthread_barrier_destroy(&threadBarrier);

  // If something is recieved, the client found the right key.
  // Now tell all other clients to stop running.
  const char* endMessage = "STOP";
  for (unsigned short i = 0; i < connectSockFDs.size(); i++)
  {
    // Stop all Client programs by sending them a message.
    int sendResult = send(connectSockFDs.at(i), endMessage, sizeof(endMessage), 0);
    if (sendResult == -1)
    {
      for (int j = 0; j <= i; j++)
      {
        close(serverSockFDs.at(j));
        close(connectSockFDs.at(j));
      }
      cout << "Issue telling client #" << i << " to stop executing." << endl;
      perror("Stopping client failed.");
      exit(EXIT_FAILURE);
    }
  }

  for (unsigned int i = 0; i < serverSockFDs.size(); i++)
  {
    close(serverSockFDs.at(i));
    close(connectSockFDs.at(i));
  }
  return EXIT_SUCCESS;
}

\end{verbatim}



\subsection{client.cpp}
\begin{verbatim}
#include <cstring>
#include <sys/socket.h>
#include <netinet/in.h>
#include <unistd.h>
#include <arpa/inet.h>
#include <cmath>
#include <pthread.h>
#include <iostream>
#include <fstream>
#include <bitset>
#include <limits.h>

#include "DESCipher.h"
#include "DESBreakConsts.h"
using namespace std;

const unsigned long TOTAL_KEYS = ULONG_MAX;
const short NUM_THREADS = 12;
const int TOTAL_THREADS = NUM_THREADS * CLIENT_COUNT;
const unsigned long THREAD_KEY_OFFSET = TOTAL_KEYS/TOTAL_THREADS;
// Have threads explore a few more keys than they're assigned to account for
// mathematical errors.
const int BITS_IN_KEY = 64;

struct thread_data
{
   int threadId;
   int clientId;
   unsigned long startingKeyNum;
   int cliSockFileDesc;
   vector<char> cipherText;
   string plainText;
};

void *ThreadDecrypt(void *threadArg)
{
   struct thread_data *threadData;
   threadData = (struct thread_data *) threadArg;

   DESCipher cipher;
   string decryptResults;
   vector<char> cipherText = threadData->cipherText;
   vector<bool> keyBits;
   unsigned int startingKeyNum = threadData->startingKeyNum;

   //Number of character groups that will need to be decrypted
   short charGroupCount = cipherText.size()/CHARS_IN_BLOCK;
   unsigned long threadKeyRange;

   if (threadData->clientId == (CLIENT_COUNT - 1) &&
       threadData->threadId == (NUM_THREADS - 1))
   {
      threadKeyRange = ULONG_MAX - startingKeyNum;
   }
   else
   {
      threadKeyRange = THREAD_KEY_OFFSET;
   }
   unsigned long endingKeyNum = startingKeyNum + threadKeyRange;
   cout << "Client: " << threadData->clientId << endl;
   cout << "Thread: " << threadData->threadId << endl;
   cout << "  Key range: " << threadKeyRange << endl;
   cout << "  Space Min: " << startingKeyNum << endl;
   cout << "  Space Max: " << endingKeyNum << endl;

   unsigned long currentKey;
   for (unsigned long i = 0; i < threadKeyRange; i++)
   {
      decryptResults = "";
      currentKey = startingKeyNum + i;
      //cout << endl << "Thread ID:  " << threadData->threadId << "  " <<
      //" Starting Key Number : " << startingKeyNum << "    " <<
      //" Current Key Number : " << startingKeyNum + i << "    " << endl;
      bitset<BITS_IN_KEY> keyBitset (currentKey);
      //cout << "keyBits: " << keyBitset << endl;
      for(short j = BITS_IN_KEY - 1; j >= 0; j--)
      {
        keyBits.push_back(keyBitset[j]);
      }

      for (int j = 0; j < charGroupCount; j++)
      {
        //Holds characters of the current group
        vector<char> curGroupChars;
        //Current character group represented as bits
        vector<bool> curCharGroup;
        //Loads 8 characters into curGroupChars
        for (short k = 0; k < CHARS_IN_BLOCK; k++)
        {
          curGroupChars.push_back(cipherText.at((j*CHARS_IN_BLOCK)+k));
        }
        //Converts current group to bit representation
        curCharGroup = DESCipher::charsToBits(curGroupChars);
        //Encrypts current group and appends output plaintext to plainText

        decryptResults += cipher.decrypt(curCharGroup, keyBits);
      }
      if (decryptResults.compare(threadData->plainText) == 0)
      {
        const char* foundKey = to_string(currentKey).c_str();
        send(threadData->cliSockFileDesc, foundKey, sizeof(foundKey), 0);
      }
   }
   pthread_exit(NULL);
}

void parentMethod(int clientId, int cliSockFileDesc, vector<char> cipherText,
                  string plainText)
{
   pthread_t threads[NUM_THREADS];
   struct thread_data threadData[NUM_THREADS];

   // Client gives the ID - the order in which client connected to server
   // The ID acts as a multiplier(0-11)
   unsigned long parentKeySpaceStart = (TOTAL_KEYS / CLIENT_COUNT) * clientId;
   cout << endl << "parentKeySpaceStart:  " << parentKeySpaceStart << endl;

   for( int i = 0; i < NUM_THREADS; i++ )
   {
      cout <<"main() : creating thread, " << i << endl;
      threadData[i].threadId = i;
      threadData[i].clientId = clientId;
      threadData[i].startingKeyNum = parentKeySpaceStart + (i * THREAD_KEY_OFFSET);
      threadData[i].cliSockFileDesc = cliSockFileDesc;
      threadData[i].cipherText = cipherText;
      threadData[i].plainText = plainText;

      cout << "i: " << i << "  |  parentKeySpaceStart + " <<
         "(i * THREAD_KEY_OFFSET): " << parentKeySpaceStart +
         (i * THREAD_KEY_OFFSET) << endl;

      int resultCode = pthread_create(&threads[i], NULL, ThreadDecrypt, (
         void *)&threadData[i]);

      if (resultCode)
      {
         cout << "Error:unable to create thread," << resultCode << endl;
         exit(-1);
      }

   }
   char recMsg[MAX_LINE];
   recv(cliSockFileDesc, recMsg, MAX_LINE, 0);
   cout << recMsg << endl;
}

vector<char> readInputAsVector(string inFileName)
{
  ifstream inTextFileStream;
  inTextFileStream.open(inFileName, std::ios::binary);
  //Vector to hold read characters
  vector<char> readText;
  //If file was succesfully opened
  if (inTextFileStream.is_open())
  {
    char c;
    //Pushes current char to readText
    while (inTextFileStream.get(c))
    {
      readText.push_back(c);
    }
    inTextFileStream.close();
    //Gets number of characters that must be padded
    short charsToPad = CHARS_IN_BLOCK - (readText.size() % CHARS_IN_BLOCK);
    if (charsToPad != CHARS_IN_BLOCK)
    {
      for(short i = 0; i < charsToPad; i++)
      {
        readText.push_back('x');
      }
    }
  }
  //File could not be opened
  else
  {
    cout << "The file " << inFileName << "was not able to be opened." << endl;
    inTextFileStream.close();
  }
  return readText;
}

string readInputAsString(string inFileName)
{
  ifstream inTextFileStream;
  inTextFileStream.open(inFileName, std::ios::binary);
  //Vector to hold read characters
  //string readText;
  //If file was succesfully opened
  if (inTextFileStream.is_open())
  {
    string readText((istreambuf_iterator<char>(inTextFileStream)),
                     istreambuf_iterator<char>());
    inTextFileStream.close();
    //Gets number of characters that must be padded
    short charsToPad = CHARS_IN_BLOCK - (readText.size() % CHARS_IN_BLOCK);
    if (charsToPad != CHARS_IN_BLOCK)
    {
      for(short i = 0; i < charsToPad; i++)
      {
        readText.push_back('x');
      }
    }
    return readText;
  }
  //File could not be opened
  else
  {
    cout << "The file " << inFileName << "was not able to be opened." << endl;
    inTextFileStream.close();
    return "File reading failed.";
  }
}

int main()
{
  string cipherTextFileName = "ct.txt";
  string plainTextFileName = "pt.txt";
  vector<char> cipherText = readInputAsVector(cipherTextFileName);
  string plainText = readInputAsString(plainTextFileName);

  string targetIP;
  /*ONE MUST BE UNCOMMENTED FOR FUNCTIONING PROGRAM
  BOTTOM ONE IS FOR LOCALHOST TESTING*/
  //targetIP = "150.243.146.141";
  targetIP = "127.0.0.1";
  int cliSockFileDesc;

  struct sockaddr_in clientAddress;
  memset(&clientAddress, 0, sizeof(clientAddress));
  clientAddress.sin_family = AF_INET;
  clientAddress.sin_addr.s_addr = inet_addr(targetIP.c_str());
  clientAddress.sin_port =  htons(PORT_NO);
  char recMsg[MAX_LINE];
  int clientId;

  // AF_INET = Use IPv4,  SOCK_STREAM = Use TCP
  cliSockFileDesc = socket(AF_INET, SOCK_STREAM, 0);
  if (cliSockFileDesc == -1)
  {
    perror("Socket descriptor creation failed.");
    exit(EXIT_FAILURE);
  }

  //Connection of the client to the socket
  if (connect(cliSockFileDesc, (struct sockaddr *) &clientAddress,
    sizeof(clientAddress)) == -1)
  {
    close(cliSockFileDesc);
    perror("Problem in connecting to the server");
    exit(EXIT_FAILURE);
  }

  int receiveResult = recv(cliSockFileDesc, recMsg, MAX_LINE, 0);
  if (receiveResult == 0)
  {
    cout << "No available messages, or connection gracefully closed. " << endl;
  }
  else if (receiveResult == -1)
  {
    perror("Receiving message failed.");
    exit(EXIT_FAILURE);
  }
  else
  {
    clientId = atoi(recMsg);
    parentMethod(clientId, cliSockFileDesc, cipherText, plainText);
  }

  close(cliSockFileDesc);
  return EXIT_SUCCESS;
}

\end{verbatim}
\end{document}
